\chapter{Abstrakt}

\authorAbstract: Nástroj na interaktívne experimentovanie s umelými neurónovými sieťami\ [Diplomová práca]. \university. \faculty; \educationalDepartment. Vedúci diplomovej práce: \advisor, \place: \facultyShort\ \universityShort, \date. \ref{TotPages}~s.

\bigskip

Táto práca sa zaoberá biologicky-inšpirovaným výpočtovým modelom -- umelými neurónovými sieťami. Argumentujeme, že hlboké neurónové siete sú perspektívny výpočtový model a poukazujeme na fakt, že jeho využitie je silne obmedzené nástrojmi, ktoré poskytujú túto funkcionalitu. Tvrdíme, že treba nové nástroje, ktoré nám pomôžu vytvárať a porozumieť systémom, ktoré sú založené na umelých neurónových sieťach. Popisujeme tri princípy, ktoré by mali byť dodržané pri vytváraní takéhoto prostredia a predstavujeme prototyp, v ktorom sú tieto princípy implementované.

\bigskip

\noindent Kľúčové slová: strojové učenie, hlboké neurónové siete, vizuálne programovanie

\chapter*{Abstract}

\authorAbstract: Interactive Experimentation Tool for Artificial Neural Networks [Master Thesis]. Comenius University in Bratislava. Faculty of Mathematics, Physics and Informatics; Centre for Cognitive Science. Master Thesis Supervisor: \advisor, \place: \facultyShort\ \universityShort, \date. \ref{TotPages}~p.

\bigskip

This thesis deals with biologically-inspired computational model -- artificial neural networks. We argue that deep neural networks are a prospective computational model, and we point out that its application is heavily limited due to tools, which provide this functionality. We claim there is a need for new tools which can help us create and understand systems based on the neural networks. We define three principles, which should be adhered to when conceiving such environment, and we present a prototype which implements these principles.

\bigskip

\noindent Keywords: Machine Learning, Deep Neural Networks, Visual Programming