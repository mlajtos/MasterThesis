\chapter{Zadanie}

\section*{Anotácia}
Umelé neurónové siete sú jedným z najčastejšie používaným modelovacím nástrojom v kognitívnej vede. Patria medzi povinnú literatúru aj v doméne umelej inteligencie. Často ich však ľudia považujú za niečo príliš zložité až mysteriózne, pretože pre ich kompletné porozumenie je potrebné ovládať zložitú matematiku a pre ich implementáciu zas programovanie na určitej úrovni.

Napriek tomu, že v modernej kognitívnej vede dominuje prístup spoznávania pomocou skúsenosti, paleta grafických demonštračných nástrojov na výuku neurónových sietí je stále pomerne chudobná. Ideálnym médiom pre vytvorenie platformy pre demonštráciu neurónových sietí je teraz už všade dostupný web. S pokrokom webových technológií a programovacích jazykov prichádza aj dobrá možnosť pre vznik výučbového prostredia pre online vytváranie a trénovanie neurónových sietí.

\section*{Cieľ}
V teoretickej časti práce, študent priblíži model umelého neurónu a typy umelých neurónových sietí, v ktorých sa takýto neurón používa. V súlade s nadobudnutými poznatkami z kognitívnej psychológie, študent navrhne softvérové prostredie, v ktorom možno interaktívne experimentovať s takýmito sieťami. Následne v praktickej časti práce, študent naimplementuje navrhnutý softvérový nástroj schopný demonštrácie rôznych typov sietí na vybranom probléme.

\section*{Literatúra}

\begin{enumerate}
  \item \cite{Rojas96}
  \item \cite{Johnson2010}
  \item \cite{Victor2013}
\end{enumerate}